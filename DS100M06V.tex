\documentclass[11pt]{article}

%\usepackage{sectsty}
\usepackage{siunitx}
\usepackage{tabularx}
\usepackage{float}
\usepackage{graphicx}
\usepackage{mathrsfs}
\usepackage{subcaption}
\usepackage{hyperref}
\usepackage{url}
\usepackage{csquotes}
\usepackage{verbatim}
\usepackage{cite}
\usepackage{stfloats}
\usepackage{textcomp}
\usepackage{algorithm}
\usepackage{algorithmic}
\usepackage{amsmath, amsfonts}
\usepackage{cmsrb}
\usepackage[serbian]{babel}

% Margins
\topmargin=-0.45in
\evensidemargin=0in
\oddsidemargin=0in
\textwidth=6.5in
\textheight=9.0in
\headsep=0.25in


\begin{document}

\tableofcontents
\pagebreak

%--Paper--
\section{Offset results}
\subsection{10x gain enabled}
-8.5 \si{\milli \volt}

\subsection{1x gain enabled}
9.2 \si{\milli \volt}

\subsection{solving}

\begin{equation}
  V_{off\_10} = 10 \cdot (2 V_{in\_off} + V_{curr\_off}) + V_{out\_off}
\end{equation}

\begin{equation}
  V_{off\_1} = 2 V_{in\_off} + V_{curr\_off} + V_{out\_off}
\end{equation}

\begin{equation}
  V_{off\_10} - V_{off\_1}  = 9 \cdot (2 V_{in\_off} + V_{curr\_off}) = -17.7 \si{\milli \volt}
\end{equation}

Total input offset is -2 \si{\milli \volt}.
Total output offset is 11.2 \si{\milli \volt}.


\section{Sources of offset}
\subsection{Input buffer}
\subsubsection{Input bias current offset}
AD8039 has an input bias current offset of $25 \si{\nano \ampere}$, across the 20 \si{\kilo \ohm}
input impedance, generates 0.5 mV of offset. Taking into account the second input buffer, maximimum offset 
is 1 mV.

\subsubsection{Input voltage offset}
AD8039 has an max input voltage offset of $3 \si{\milli \volt}$.
Worst case total offset is $6 \si{\milli \volt}$.

\subsection{Output buffer}
\subsubsection{Input bias current offset}
AD8009 has an max input current offset of $150 \si{\micro \ampere}$. With an input impedance of 
100 \si{\ohm}, total offset is 15 mV. If both inputs have oposing offsets, the total offset is 30 mV

\subsubsection{Input voltage offset}
AD8009 has an max input voltage offset of $5 \si{\milli \volt}$.

\section{Noise}

\subsection{Resistor noise}


\begin{equation}
  V_{rms} = \sqrt{4 k_B T R \Delta f}
\end{equation}

At the bandwidth of 100 \si{\mega \hertz} and temperature of 25 \si{\celsius}, noise is

\begin{equation}
  V_{rms} = 3.63 \si{\milli \volt_{RMS}}
\end{equation}

\subsection{Input buffer noise}

AD8039 has an input noise level of $8 \si{\nano \volt} / \sqrt{\si{\hertz}}$ meaning 
that total input noise rms at 100 \si{\mega \hertz} bandwidth is 80 \si{\micro \volt_{RMS}}.

\subsection{Output buffer noise}
AD8009 has an input noise level of $1.9 \si{\nano \volt} / \sqrt{\si{\hertz}}$ meaning 
that total input noise rms at 100 \si{\mega \hertz} bandwidth is 19 \si{\micro \volt_{RMS}}.


\end{document}
